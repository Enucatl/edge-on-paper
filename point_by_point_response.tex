%        File: point_by_point_response.tex Created: Wed Oct 23 05:00 PM 2013
%        C Last Change: Wed Oct 23 05:00 PM 2013 C
%
\documentclass[a4paper,english]{scrartcl} \usepackage[detect-all]{siunitx}
\usepackage{babel} \usepackage[T1]{fontenc} \usepackage[utf8]{inputenc}

\DeclareSIUnit{\voltpeak}{Vp} \newcommand{\reviewer}[1]{\emph{#1}}
\newenvironment{reviewerquote}{\begin{quote}\itshape}{\end{quote}}

\title{Response to reviewer \#3} \author{Revised manuscript SREP-13-06290}
\date{} \begin{document} \maketitle

\noindent

Reply to Review 3
=================

We would like to thank reviewer 3 for his/her detailed feedback on our
latest revision. We actually share most of his/her concerns but we also see
that there have been some misunderstanding on our side on the main points
raised by his/her latest comments. We would like to address them in the
following. 

Referee 3: "It is disappointing that the authors misinterpreted the
questions and the arguments this referee raised against the major claims of
this work. This referee did not say: "there is no need to increase the
energy to 100 keV in order to translate this technology to clinic" but
rather question that is not the major factor. This referee did not
"acknowledge(ing) that an increase to 60 keV would be good for mammography",
even though it could be. The main point of the argument is if the
contribution made by this work important or significant. The conclusion of
the first review was not enough for a journal cover all different areas of
basic science. It is probably not even sufficient to attract to readers in
the same research area of X-ray imaging. The authors reply did not change
this concern by quoting an "improvement" of increasing photon energy to 60
keV for mammography, or to higher photon energy in general for radiology in
general, a well-established and widely implemented for human health care for
decades."

Our answer: “Apparently, we have mis-interpreted the message of Reviewer 3
and we thank him/her for his effort to clarify his/her opinion. In his/her
first part of the argumentations, Reviewer 3 questions whether our
contribution is "important of significant". This is a remark which in
principle can be made to any scientific contribution, independently to its
degree of novelty, difficulty or outreach. It has a strong subjective
component, as demonstrated by the fact that two others reviewers found our
work more than worth to be published in SciRep while Reviewer 3 still has
several concerns. Therefore, we would like to reiterate here that our work
is a step towards the implementation of phase contrast imaging on clinical
systems which we believe might contribute to improve image contrast while
keeping the dose under reasonable control. Those systems are compact and are
operated at high-energies, or at least at energies much higher than those
that have been used for any phase-contrast experiment so far (to the best of
our knowledge). We are not claiming that we found the ultimate solution to
it, but we present an experimental solution to address two major concerns
(high-energy and compactness) which are presently limiting the broad
application of phase contrast imaging outside the synchrotron community.” 

Referee 3: "Specifically, this referee questioned if the photon energy (not
being high enough) and the distance (not being short enough) between
detector and specimen, are the key barriers to the translation of grating
base phase contrast radiology to clinic. The major barrier prevent the phase
contrast imaging methods, re-invented more than 20 years ago with the birth
of new synchrotron facility, to be transfer away from synchrotron and to
make an impact to clinical application at this moment, is the long exposure
time required to obtained the phase information by either high quality
sources (in-line) or treatment to the exit waves. Talbot-Lau method promises
phase contrast using conventional X-ray sources by slotting the large source
into small ones which could potentially enable the phase retrieval. However,
the requirement of scanning the gratings, in the case reported here (24
steps), inevitably increases the dose as well as the image acquisition time
which put this technique only in the research laboratory at the moment." 

Our answer: (*Still need to work on this…any suggestions welcomed*)

Referee 3: “This referee knows and agree that the radiation dose is critical
for medical application, but that was exactly where the usefulness of this
work is quested. It is well accepted that no new X-ray imaging technology is
likely be implemented for clinic with dose penalties. There is even more
fundamental question on if the phase information can really benefit clinical
diagnosis even at the same dose. Instead of tackling the main challenges on
reducing the scanning time or on demonstrating that phase contrast indeed
give better visibility at the same dose than those without phase contrast on
critical applications such as for medical imaging, device inspection and
security inspection..., 

Our reply: "Clinical relevance of phase contrast is a fundamental question
and we addressed this in several works, specifically on mammography,
published elsewhere (InvRad 1, InvRad 2, NatComm). We are very well aware
about the importance of assessing the advantages of phase contrast imaging
with respect to existing diagnostic procedures, in particular when dealing
with dose issues. However, the main subject of the present manuscript is
different and addresses a very practical aspect: we will not even be able to
discuss about clinical relevance of phase contrast if we cannot operate
phase contrast in a clinical environment at all! With our work we want to
show a possible solution to implement phase contrast imaging in clinics and
addressing high energies and compact systems seem to us to be two
fundamental points. 

Referee 3: "…suggested secondary problems, such as photon energy and the
total dimension requirement for the instrument are targeted. The efforts are
trying to "improve", if it is indeed an improvement, a technology (the
grating base phase contrast radiology) that is interesting only for its
promised clinical application which otherwise offers no fundamental
innovation for the phase retrieval scheme in X-ray imaging (it took
advantage of a known theory in optics to shorter wavelength).”

Our reply “We do not believe that photon energy and total dimension
requirements are secondary problems. As mentioned before, the penetration
power of the probing radiation is rather a primary issue as we need
sufficient photon statistics to be generated after penetrating a whole human
body in order to measure a signal and, this, independently from whether we
do absorption or phase contrast imaging. The comments on whether the phase
retrieval procedure is innovative or not could actually be extended to the
whole field of grating interferometry, as the theory in optics is known
since the 19th century, although at the time it was only applied to longer
wavelengths. We would like to point out again that the promised clinical
applications are coming true, and that the realization of this experiment
had its remarkable challenges, despite being known in theory.”

Referee 3: “In the long list of "major players" and with all their effort
contributing to the international conferences, is there any one ready to
commit this technology as their product roadmaps? This work therefore
promises a potential benefit on a potential technique which is certainly
valuable from the instrumentation point of view and not likely to generate
much interest of readers of Scientific Reports.”

Our reply: "While most of the industrial research and product development is
obviously unknown to the general public, at least two products roadmaps have
been presented in the last months. An upgrade of the Philips MicroDose
mammography system into a Talbot-Lau interferometer was shown by the company
at the SPIE 2014 conference [2] (please add here the reference mentioned by
Zhentian). Moreover, clinical trials are currently ongoing on patients and
healthy volunteers in Japan (*give here the name of the hospital, from XNPIG
proceedings*) with a Talbot-Lau interferometer developed as a collaboration
between the University of Tohoku and Konica Minolta [1]. This last machine
operates with an acceleration voltage of 40 kVp and with an acceptable
exposure time (19 s) and dose (under 10 mGy). We think that this is
convincing evidence that this technique already belongs to the hospitals and
not only to the research laboratories, as well as being a hot topic of
recent industrial and scientific development and interest. However, the
reviewer is right when he points out that the work presented in our
manuscript is still far from applications. At the same time, we think he or
she can agree on the fact that sources with 100-120 keV are needed in
hospitals for tomography and even higher voltages are required for security
inspections or material sciences investigations. This is why we think that,
besides reduction of dose and exposure times for existing setups, the
exploration of a new energy range is a fundamental research topic. Indeed,
it is worth to mention that to now, no system in the world was able to
produce images at energies around 100 keV. Even with a high dose or long
exposure this was simply impossible because of the unavailability of the
optical elements, which we could overcome in our work. Finally, low-energy
grating interferometry is in fact making its appearance in hospitals in
dedicated areas (cartilage and mammography), and developing high-energy
systems is key to expanding its applications, as well as potentially giving
additional benefits in terms of phase contrast and dose reduction as shown
in the manuscript. This study is of course a first step toward this goal
and, while we could claim a successful design and operation of the machine,
we cannot conclude at this stage that we can have a high imaging quality
with clinically compatible exposures. (*Matteo: do you agree with this
latest sentence? You wrote “we could conclude…” but I think this was
wrong…”*).

Referee 3: "Although Scientific Reports requests evaluation on the
scientific rigorousness, over claiming the potential on clinical, device and
homeland security applications is not rigorous. The hastiness of the
presentation does not make their work rigorous, either. There is no analysis
presented on the only image result, other than just pointing out the
differential phase effect and merging images of different contrast. The
authors argued that the reduced visibility in the attenuation contrast is
due to the stronger X-ray hardening effect than that of phase contrast. This
cannot be correctly evaluated by the referees and readers without knowing
the size (along to the beam direction) of the structure of the chip,
including the solders and resistors, and the absolute gray scale of those
images. Without giving detailed analysis of the contrast on a set of poor
quality images, and perhaps compare them with lower, say 80 keV photons, it
is not convincing that the work does indeed improve the imaging quality of
visibility for their targeted applications.”

Our reply: (*This is difficult. I do not know how to rebuttal this, beside
telling that going from 100 to 80 is not simple (new gratings, different
distances, complicated alignment…) Any suggestion??*)

Referee 3: “There are also other flaws of this presentation, which were not
pointed out in the first round reviewer's report. First of all, there is no
mention about the exact photon energy or range actually used. Referring to
manufactures data and provide a few more words than "designed energy of 100
keV" would be more quantitative. The accelerating energy and photon energy
are used in a confusion fashion of kV, kVp and keV for non-specialists. If
the average photon energy is ~100 keV then the author must demonstrate
significant difference in terms of the image quality between previous
reports of 82keV and 60 keV using synchrotron and Talbot-Lau methods. If
photon energy is one of the major claim of advantage of this paper, the
authors cannot escape this comparison by just saying they are done with
synchrotron or by other type of grating arrangements”

Our reply: "More details on the source, such as the material of the anode
and the current have been added to the manuscript. The energy range is not
changed afterwards by any means, so this is indeed the full information
needed to reproduce the experiment on any similar generator. The design
energy is a property of the interferometer, and is defined in the manuscript
as the energy at which “the beam splitter grating periodically shifts the
phase by zero and Pi”. We suggest that this is already a precise
statement. Concerning the comparison with synchrotron experiments, we think
that at this stage of the developments, it should be sufficient to show that
we can generate phase contrast images at 100 keV on a conventional source
and to provide the discussion the way and extend we did. The differences in
flux, coherence and monochromaticity between a synchrotron- and a tube-
based experiment are so huge that the fact already that we were able to
obtain phase image should be considered as a success and worth to be
published. Again, to now, nobody was able to achieve this. And, it is not
just “an other type of grating arrangement”. We tried to explain that our
suggestion  allows to perform phase contrast imaging at really high
energies, i.e., this might be “THE” arrangement.

Referee 3: “The claim of potential having an order of magnitude decrease in
the scanning time is not supported by their own argument. If it is indeed
limited by the visibility of the grating, how can an improvement from 5%
(device reported here) to 14% (current best technology) reaching one order
of magnitude, "easily”? 

Our reply: “This is because, as explained in the methods section, some
smaller section of our current gratings already have a visibility of 14 %.
This means it is almost three times as good as the average, and since the
signal-to-noise ratio drops with the square root of the exposure time, but
proportionally to the visibility, we claim that improving the visibility by
a factor of three allows us to shorten the exposure time by a factor of nine
(*As Zhentian suggested, please add here a reference..."

Referee 3: "Even with one order of magnitude improvement, could they obtain
an image with scanning at a speed fast enough to match the desired
motion-free imaging? “

Our reply: "The Reviewer should agree that if we were already able to do
this, the focus and the cut of our publication would be completely
different. Namely, we would be already there were we are aiming at: phase
contrast imaging at high-energy, on compact systems, fast and, ideally, with
same or lower dose. We can only hypothesize that, with the suggested grating
design, there are no indications against reaching the theoretical maximal
visibility one might expect at 100 keV. This would be physically the best
result achievable. If, in that case, the hardware around the gratings will
be carefully selected, we do not see why we should not be able to acquire
tomographic slices with speed as present, absorption-based systems." 

Referee 3: "The authors argued (in their reply) that they can increase the
acquisition efficiency by stacking the grating. In this case, there will be
200 such devices needed to be stacked to cover a 2 cm wide area, which they
have to consider the curvature in this direction as well. At the same time
they argued that scanning a small beam (a line illumination in this case) is
better than full field imaging. They quote a commercial system claiming
reduction of dose which is likely, as pointed out by the author, due to the
strongly collimation, therefore, is irrelevant to the comparison.”

Our reply: The dose argument between scanning, highly collimated systems
(Philips Microdose) versus others, based on full-area imaging, is not
irrelevant, as this is one of the major selling arguments for the low-dose
Philips systems. Therefore, if we combine a series of collimated, fan beams
with our gratings geometry and scan them to cover a larger area, we will
indeed have less dose delivered to the specimen/patient compared to a
cone-beam approach, requiring a full illumination of the sample. Covering a
2cm wide area in one direction would indeed be a challenge, we agree. The
question is whether we would need to really cover a 2cm wide area or if we
could obtain acceptable results with less (while scanning longer). We also
have to consider the size of the pixel we are aiming for, which might be
slightly larger than 100 microns, at least for a human CT system.

Referee 3: “The fabrication of the grating is not described specifically,
other than saying "Grating design and fabrication is nonstandard and
involves a complex mask design" with a reference to a fabrication company.
It is not likely one can follow the actual process with just these
information. 

Our reply: “The fabrication through lithography and electroplating is not
the main topic of this manuscript, hence it is only cited in 22, 23, 24. The
mask design was carried out by MicroWorks GmbH, but it is a very technical
detail that we think bears no impact on the reproducibility of the
experiment: any design that reliably builds the described circularly
arranged pattern will produce usable gratings"

Referee 3: "The SEM image on a selected portion of a grating already shows a
rather poor quality placement of um size structures, no mention about how
the precision of these structure could affect the imaging or the phase
retrieval."

Our reply: “We mention in the manuscript that these irregularities are
responsible for the limited visibility of the interferometer, and the
connection of this value to the signal-to-noise ratio in the images."

Referee 3: “In terms of the innovation in the arrangement the grating in the
"edge-on" fashion, Ref. 25 indeed reported a similar arrangement to overcome
the aspect ratio limitation. As to the adoption of a curved structures, it
was also proposed in the fabrication of multilayer Laue lens to deal with
the similar problem by tilting, wedging or curving the nanostructures.
Therefore, the work indeed puts these ingredients together to realize the
device, but does not reach high innovation level by applying it to a device
at different, actually easier, length scale"

Our reply: “We do not claim to have reached a high-innovation level with our
suggested grating arrangement. But we present  phase contrast images at 100
keV on a compact system. To the best of our knowledge this is novel and, as
we tried to explain, could have a huge impact.”

Referee 3: “With these observations, this referee argues that this paper
does not meet the high scientific or innovation value of Scientific Reports.
The presentation of the results is not rigorous which is not likely be
corrected without much more investigation to build a better publishable
case. The rebuttal did not address the main concern of the last review,
therefore, this referee does not consider this work after revision
publishable for Scientific Reports.”

Our reply: “We hope to have addressed in our present reply the major
concerns of the reviewer. We tried to clarify where we see our work to be
novel and where we discuss the potential benefits of our approach. The
generation of a phase contrast image at 100 keV using a conventional X-ray
tube in a compact setup is something which was never done before and that
intrinsically carries a high potential for medical applications and beyond.
This is our message and we are convinced of its relevance and impact which
seems to be well matched to the requests of Scientific Reports a journal
which, to the best of our knowledge, has no absolute novelty criterion as a
pre-requisited for a manuscript to be eligible for publication."
\end{document}
