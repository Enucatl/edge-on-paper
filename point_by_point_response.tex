%        File: point_by_point_response.tex
%     Created: Wed Oct 23 05:00 PM 2013 C
% Last Change: Wed Oct 23 05:00 PM 2013 C
%
\documentclass[a4paper,english]{scrartcl}
\usepackage[detect-all]{siunitx}
\usepackage{babel}
\usepackage[T1]{fontenc}
\usepackage[utf8]{inputenc} %Umlaute

\DeclareSIUnit{\voltpeak}{Vp}
\newenvironment{reviewerquote}{\begin{quote}\itshape}{\end{quote}}

\title{Point-by-point response to the reviewers}
\author{Revised manuscript NCOMMS-13-09247-T}
\date{}
\begin{document}
\maketitle

\noindent
The comments from the reviewers are quoted in \emph{italic}, whereas the
revised article parts are reported in upright font in indented sections.
\section*{Reviewer \#1}
\begin{reviewerquote}
    I could not find a significant advance in this work. X-ray Talbot-Lau
    interferometers with high-energy x-rays have been already reported (Ref.
    18-20, 23, D. Stutman et al., Appl. Opt. 49 (2010) 4677 etc.). The
    authors mention that the edge-on- illuminated grating interferometry
    breaks the current limitations of x-ray phase-contrast imaging, but I
    don't think that there is clear evidence showing that the authors'
    method is much more promising than the methods reported in the previous
    papers. If the authors want to show that the edge-on-illumination method
    is by far the most promising for high-energy and large-field-of-view
    x-ray phase imaging, I think that the authors have to quantitatively
    compare the method with those previously reported in terms of the total
    exposure time, spatial resolution, signal-to-noise ratio, and so on, and
    show that the authors' method is much better than the others.
\end{reviewerquote}

Stuntman et al.  present a method which increases the effective
aspect ratio of the gratings by slightly rotating them along an axis
perpendicular to the propagation direction. However, his method has
the major limitation of not being compatible with divergent
geometries. As a result, his interferometer will show a high
visibility, although still limited compared to what we can achieve in
theory, but its horizontal field of view will be strongly limited in
case of divergent beams, as commonly found in any table-top device.
Our solution, on the other hand, solves both issues at the same time,
and therefore we consider this as a significant improvement with
respect to Stuntman's work.  Moreover, the cited experiments were
performed on a source with~\SI{60}{\kilo\voltpeak}, which is significantly lower than
the~\SI{160}{\kilo\voltpeak} of our source and the~\SI{100}{\kilo\eV} used as the design energy
for our interferometer. Reviewer \#1 further cites other related works
which, however, where carried out either in much lower energy range or
under very different imaging conditions, thus preventing a meaningful
comparison. 

Reference~23 reported images taken at \num{30} and~\SI{60}{\kilo\eV},
reference~20 also has a design energy of~\SI{47.9}{\kilo\eV}. Finally,
references~18 and~19 present Talbot interferometry at~\num{82}
and~\SI{123}{\kilo\eV} respectively, but only the first result is published
on a peer reviewed journal, and both experiments were performed at the ESRF
synchrotron, with a monochromatic and coherent beam that is incomparable to
a conventional tube.
In order to emphasize this enormous difference between synchrotron sources
and lab sources, the title has been changed to ``X-ray phase-contrast imaging at
100 keV on
a conventional source''.

In order to address the concerns of reviewer \#1, the short review of the
previous results in the
introductory paragraph has been updated:
\begin{quote}
The vast majority of phase-sensitive techniques, including crystal analyzer based [5, 6] or
interferometric [7, 8] methods rely on X-ray beams of high spatial and temporal coherence,
which is available only at synchrotron sources. Inline phase contrast [9–11] and Talbot interferometry [12–14] need high spatial coherence but are available on polychromatic microfocus
sources. Phase-contrast imaging using X-ray beams of low temporal and spatial coherence
such as conventional low-brilliance X-ray tubes have been demonstrated with coded apertures [15] and Talbot-Lau interferometry [16]. Analyzer-based systems have been recently
extended to tube sources [17, 18] but only at energies up to the tungsten
$K_\alpha$ line at 60 keV. In
addition to phase sensitivity, analyzer-based and Talbot interferometry also provide (with different retrieval mechanisms) information about the integrated
local small angle scattering power from microscopic density fluctuations in a specimen [19].
This signal is known under the name of dark-field, scatter or visibility reduction contrast.
High-energy Talbot interferometry has been reported so far using a synchrotron source
at nominal energies of 82 keV [20] and 123 keV [21]. Using a low-brilliance X-ray tube,
Talbot-Lau interferometry was applied so far at 60 keV mean energy [22]. Medical imaging
applications may benefit from phase contrast at higher energies: chest or abdominal radiography or CT require an acceleration voltage between 100 and 150 kVp. Other potential applications are
homeland security or chip failure analysis, which require high energies for the visualization
of materials of high density and atomic number.
\end{quote}

\begin{reviewerquote}
    Furthermore, the authors mention that the results in Fig. 3 show the benefit of the phase nature of the image, but I think that the authors should provide a more detailed explanation about the contrasts in the images because the contrast of soldering points underneath the resistors should not be reduced in the differential attenuation image for monochromatic x-rays. The authors should explain quantitatively why the contrast is reduced. Otherwise, the authors' assertion is suspicious.
\end{reviewerquote}
A paragraph has been added to discuss the different beam hardening behaviour of
the absorption and differential phase image:
\begin{quote}
In the attenuation image, the contrast of the soldering points of the integrated circuit is reduced underneath the
resistors, while in the phase image, they can clearly be identified. The reduced contrast
of the soldering points in the absorption image is due to beam hardening. The spectrum
impinging on these soldering point is hardened by the resistors in the upper layer, resulting
in lower absorption contrast. Due to the weaker energy dependency of phase
shifts ($1/E$
compared to $1/E^3$), phase-contrast images are less sensitive to beam hardening [31], which
explains the lower contrast reduction of the soldering points underneath the resistors in the
phase image of the chip. This result shows the benefit of the phase contrast in high-energy
X-ray imaging, which may be useful to identify flaws in multilayered structures such as
electronic chips.
\end{quote}

\section*{Reviewer \#2}
\begin{reviewerquote}
    However, this solution comes with significant drawbacks in terms of very
    limited fields of view~\SI{3}{\centi\metre} horizontally, a single detector pixel in the vertical direction), which at the moment seem hard to overcome.
\end{reviewerquote}
There is no fundamental limitation on the horizontal field of view in our
approach. On the contrary, our solution actually solves precisely this
issue, by easily matching the spherical wave front with the curved
structures.
Our approach removes any limitation to the horizontal field of view which would arise due to the large beam divergence. In addition, curved gratings can be easily stitched to cover an even larger field of view, if necessary.

With our scanning solution, the limitation of the vertical field of view does not come at the expense of
an increased dose. 
A similar scanning technique is already commercially offered in X-ray
mammography systems, like the MicroDose instrument of Philips, with a
significant reduction of the scattered radiation. We discussed this issue and
amended the text accordingly.
\begin{quote}
 Radiographic 2D imaging can be obtained by scanning the sample or a
thin fan beam. The scanning technique has been demonstrated to deliver less dose than the
conventional approach based on the illumination of a large area. In digital mammography,
for instance, where dose is a critical issue, Philips’ MicroDose system combines a scanning
approach with an highly collimated fan beam [27]. Thanks to the high collimation, the dose
deposited on patients has been reported to be significantly lower than with other instruments
based on the illumination of a large area detector [28]. 
\end{quote}

\begin{reviewerquote}
    Moreover, in the methods section the authors talk of a
    \SI{5}{\percent} visibility, which is even lower than that of the quoted papers by Pfeiffer's group (refs 18, 19), which, at least nominally, reach even higher energies (123 keV). Hence, the reduction in field-of-view does not seem to be counterbalanced by significant improvements in visibility. This is also supported by the relatively poor quality of the differential phase image compared to the derivative of the absorption image, where the only advantage seems to be the visibility of the soldering points beneath the resistors (primarily due to the fact that these do not produce area contrast in the DPC image), while everywhere else the contrast of the differentiated absorption image seems to be higher.
\end{reviewerquote}
We would like to remind to the reviewer \#2 that our data have been obtained on
a commercially available X-ray tube, while the cited experiment of Pfeiffer's
et al. has been carried out at a third generation synchrotron source, where
notably unsurpassed beam conditions can be created which are far from what one
can expect in real life (i.e. with conventional X-ray tubes). And, even if one
would like to still carry out this (unfair) comparison, we should mention that
our experiment is the first measurement of its kind with the first generation
of gratings made according to our innovative design. Pfeiffer's team uses
gratings obtained after almost a decade of development and, despite this and
the use of a synchrotron, they are not really further than us while they still
carry the intrinsic limitation of a limited grating height. We are well aware
that the quality of our images has still to be improved, but here we are
presenting the first experiments with edge-on grating illumination: we describe
its great potential but also mention its (temporary) limitations.
\begin{reviewerquote}
Finally, the novelty of the result is diminished by previous demonstrations that high energies in phase-contrast x-ray imaging could be reached by other methods (e.g. Ignatyev et al JAP 110 (2011) 014906), albeit in that case image quality seems to be lower.     
\end{reviewerquote}
Reviewer \#2 admits that in comparison with a coded-aperture based experiment at 100 kVp (Ignatyev et al.) our images (taken at 160 kVp) are better, which we of course appreciate.

\section*{Reviewer \#3}

\begin{reviewerquote}
    An extended object is imaged by axial scanning, which results in very
    long exposure times due to phase stepping.  The legend of Fig. 3
    indicates that one image line of \SI{0.1}{\milli\metre} width is
    acquired in \SI{6}{\minute}.  The authors admit this situation in the last sentences of the manuscripts without giving a convincing solution to the problem.
\end{reviewerquote}
This is indeed an issue, as the exposure time is mostly affected by the low visibility. The manuscript
was amended to include a more detailed discussion of our plans to reduce
the exposure time. We mention now that raising the visibility to
\SI{15}{\percent}, a value already reached in lower energy experiments and
in some sections of our gratings,
would immediately cut the exposure by almost an order of magnitude:
\begin{quote}
The long exposure times are mostly constrained by the low average
visibility of the gratings (5 \%). The exposure time was chosen in order to get a low noise in
the differential phase image. The signal-to-noise ratio (SNR) is proportional to the visibility
and the square root of the exposure time [33]. This implies that the exposure times can
easily drop by an order of magnitude as these gratings become comparable in quality to
those developed in the last ten years. Smaller regions of these gratings actually exhibit a
visibility up to 14 \% already, indicating that this goal is reachable as the fabrication becomes
more reliable and uniform.
\end{quote}
\begin{reviewerquote}
    The authors suggest that the only imaging methods that can be used with
    conventional x-ray sources are the in-line phase contrast technique and
    Talbot interferometry. They fail to mention that analyzer-based imaging
    systems with x-ray tube sources have been built and successfully used
    for imaging rather large biological objects. The current exposure times
    are shorter or comparable to those of the present work, but there are
    clear indications how the exposure times in x-ray tube based ABI can be
    reduced by an order of magnitude or more. On the other hand, the upper
    limit of photon energy is that of the tungsten Kalpha line,
    \SI{60}{\kilo\eV}. For references, see Nesch et al.(2009), Rev. Sci.
    Instrum. 80, 093702, and Parham et al.(2009), Acad.Radiol. 16, 911-917.
\end{reviewerquote}
We thank Reviewer \#3 for pointing out this issue. We amended the text
accordingly, discussing specifically the cited analyzer-based imaging experiment. In particular,
we highlight that the mentioned ABI investigation was carried out at 60 keV,
far below our nominal energy of 100 keV. We are also not aware of any ABI
operated at energies higher or equal to 100 keV on conventional X-ray sources. 
\end{document}


