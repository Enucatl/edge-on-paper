%        File: point_by_point_response.tex
%     Created: Wed Oct 23 05:00 PM 2013 C
% Last Change: Wed Oct 23 05:00 PM 2013 C
%
\documentclass[a4paper]{scrartcl}

\title{Point-by-point response to the reviewers}
\date{}
\begin{document}
\maketitle

\section*{Reviewer \#1}
We would like to provide more information on our experiments, as recommended
by the first reviewer.
\begin{quote}
    I could not find a significant advance in this work. X-ray Talbot-Lau interferometers with high-energy x-rays have been already reported (Ref. 18-20, 23, D. Stutman et al., Appl. Opt. 49 (2010) 4677 etc.). The authors mention that the edge-on- illuminated grating interferometry breaks the current limitations of x-ray phase-contrast imaging, but I don't think that there is clear evidence showing that the authors' method is much more promising than the methods reported in the previous papers. If the authors want to show that the edge-on-illumination method is by far the most promising for high-energy and large-field-of-view x-ray phase imaging, I think that the authors have to quantitatively compare the method with those previously reported in terms of the total exposure time, spatial resolution, signal-to-noise ratio, and so on, and show that the authors' method is much better than the others.
\end{quote}
We think that the experiments presented in the current manuscript show
indeed significant progress with respect to the cited references.
The method presented by Stutman et al. has a fundamental limitation that
prevents it from achieving curved structures and large aspect ratios.
Moreover, those experiments were performed on a source
with~\SI{60}{\kilo\volt\peak}, which is significantly lower than
the~\SI{160}{\kilo\volt\peak} of our source and the~\SI{100}{\kilo\eV} used
as the design energy for our interferometer.

The other works also present results in a much lower energy range or very
different conditions, thus preventing a meaningful comparison.
Reference~23 reported images taken at \num{30} and~\SI{60}{\kilo\eV},
reference~20 also has a design energy of~\SI{47.9}{\kilo\eV}. Finally,
references~18 and~19 present Talbot interferometry at~\num{82}
and~\SI{123}{\kilo\eV} respectively, but only the first result was published
in a peer reviewed journal, and both experiments were performed at the ESRF
synchrotron, with a monochromatic and coherent beam that is incomparable to
a conventional tube.

\begin{quote}
    Furthermore, the authors mention that the results in Fig. 3 show the benefit of the phase nature of the image, but I think that the authors should provide a more detailed explanation about the contrasts in the images because the contrast of soldering points underneath the resistors should not be reduced in the differential attenuation image for monochromatic x-rays. The authors should explain quantitatively why the contrast is reduced. Otherwise, the authors' assertion is suspicious.
\end{quote}

\section*{Reviewer \#2}
\begin{quote}
    However, this solution comes with significant drawbacks in terms of very
    limited fields of view~\SI{3}{\centi\metre} horizontally, a single detector pixel in the vertical direction), which at the moment seem hard to overcome.
\end{quote}
There is no fundamental limitation on the horizontal field of view in our
approach. On the contrary, our solution actually solves precisely this
issue, by easily matching the spherical wave front with the curved
structures. Any limitation to horizontal the field of view given by large glancing
angles is therefore removed.

The limitations in the vertical field of view do not come at the expense of
an increased dose with our proposed scanning solution. Larger areas could in
principle be scanned with short exposure times by stacking an array of
edge-illuminated gratings. This is already offered in some commercial X-ray
diagnostic devices such as the Philips MicroDose mammography system.

\begin{quote}
    Moreover, in the methods section the authors talk of a
    \SI{5}{\percent} visibility, which is even lower than that of the quoted papers by Pfeiffer's group (refs 18, 19), which, at least nominally, reach even higher energies (123 keV). Hence, the reduction in field-of-view does not seem to be counterbalanced by significant improvements in visibility. This is also supported by the relatively poor quality of the differential phase image compared to the derivative of the absorption image, where the only advantage seems to be the visibility of the soldering points beneath the resistors (primarily due to the fact that these do not produce area contrast in the DPC image), while everywhere else the contrast of the differentiated absorption image seems to be higher.
\end{quote}
The visibility in our experiment is still low, and this obviously
affects the quality of the images, but we also mentioned
that this is the first attempt in building gratings with our new design,
while the fabrication of traditional structures has been developed for over
a decade now. The remarks above about references~18 and~19 (synchrotron
source, unpublished results) should also be taken into account here.
\begin{quote}
Finally, the novelty of the result is diminished by previous demonstrations that high energies in phase-contrast x-ray imaging could be reached by other methods (e.g. Ignatyev et al JAP 110 (2011) 014906), albeit in that case image quality seems to be lower.     
\end{quote}
In the paper from Ignatyev et al., an admittedly lower image quality was
achieved again in a lower energy range, from~\num{60}
to~\SI{100}{\kilo\volt\peak}. Thus, we would like to claim that our results
are a significantly more promising than the technique described therein.

\section*{Reviewer \#3}

\begin{quote}
    An extended object is imaged by axial scanning, which results in very
    long exposure times due to phase stepping.  The legend of Fig. 3
    indicates that one image line of \SI{0.1}{\milli\metre} width is
    acquired in \SI{6}{\minute}.  The authors admit this situation in the last sentences of the manuscripts without giving a convincing solution to the problem.
\end{quote}
This is indeed an issue, as the exposure time is mostly affected by the low visibility. The manuscript
was amended to include a more detailed discussion of how we plan to reduce
the exposure time. We briefly mention here that raising the visibility to
\SI{15}{\percent}, a value already reached in lower energy experiments,
would immediately cut the exposure by \num{9} times.
\begin{quote}
    The authors suggest that the only imaging methods that can be used with
    conventional x-ray sources are the in-line phase contrast technique and
    Talbot interferometry. They fail to mention that analyzer-based imaging
    systems with x-ray tube sources have been built and successfully used
    for imaging rather large biological objects. The current exposure times
    are shorter or comparable to those of the present work, but there are
    clear indications how the exposure times in x-ray tube based ABI can be
    reduced by an order of magnitude or more. On the other hand, the upper
    limit of photon energy is that of the tungsten Kalpha line,
    \SI{60}{\kilo\eV}. For references, see Nesch et al.(2009), Rev. Sci. Instrum. 80, 093702, and Parham et al.(2009), Acad.Radiol. 16, 911-917.
\end{quote}
We now included references to these papers, but we point out that our
experiments do not have any intrinsical limit to the photon energy.
\end{document}


