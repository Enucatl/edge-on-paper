%        File: point_by_point_response.tex
%     Created: Wed Oct 23 05:00 PM 2013 C
% Last Change: Wed Oct 23 05:00 PM 2013 C
%
\documentclass[a4paper,english]{scrartcl}
\usepackage[detect-all]{siunitx}
\usepackage{babel}
\usepackage[T1]{fontenc}
\usepackage[utf8]{inputenc}

\DeclareSIUnit{\voltpeak}{Vp}
\newcommand{\reviewer}[1]{\emph{#1}}
\newenvironment{reviewerquote}{\begin{quote}\itshape}{\end{quote}}

\title{Point-by-point response to the reviewers}
\author{Revised manuscript SREP-13-06290-T}
\date{}
\begin{document}
\maketitle

\noindent
The manuscript has been revised in order to address the editor's and the reviewers'
comments.
The comments from the reviewers are quoted in \emph{italic}, whereas the
revised article parts are reported in upright font in indented sections.
\section*{Reviewer \#2}
\begin{reviewerquote}
    A single wafer contains multiple gratings that are designed for a specific distance from the source. In figure 1, the designed distance from the source to each grating must be indicated, in order to make things clear for the readers. Please also explain how robust imaging quality is with mis-alignment of the grating with respect to the working distance from the source.
\end{reviewerquote}
The figures and captions have been improved in order to address the
reviewer's concerns. In particular, the distances have been added to the
caption of fig. 1:
\begin{quote}
    A \SI{100}{\kilo\eV} setup was realised where the distance
        between the source and the source grating is \SI{23}{\centi\metre}
    and the distance between the source grating and the phase grating is
    \SI{16}{\centi\metre}. That is also the distance from the phase grating
to the analyzer grating.
\end{quote}
In the text, we also included a measurement of the variation of the
visibility as the integrating distance is changed:
\begin{quote}
    The precise position of the analyzer grating along the beam axis is not
critical, and a change in visibility of less than \SI{1}{\percent} is
observed by displacing it by as much as \SI{5}{\milli\metre}.
\end{quote}
\begin{reviewerquote}
    Figure 2 shows five gratings. Based on figure 1, only three gratings are needed. Label the gratings in Figure (source grating, beam splitter grating, analyzer grating). What is the purpose of the other two gratings shown in Figure 2?
\end{reviewerquote}
The gratings in the figure have been labelled and the caption makes clear
that many sets of gratings can be fabricated simultaneously on a
single wafer:
\begin{quote}
    The top part of the 4 inch wafer shows
        five grating chips. From top to bottom, one source grating, two
        phase gratings and two analyzer gratings. The
        gratings have different curvatures which are specific to the grating
        interferometer geometry. Multiple gratings for more than one setup
        geometry are fabricated on a single wafer.
\end{quote}
\begin{reviewerquote}
    Figure 3 shows "differential" attenuation image. Please explain what this exactly is. The text is extremely vague.
\end{reviewerquote}
The derivative of the attenuation image is now described in more detail:
\begin{quote}
    For a better comparison of the magnified phase and
attenuation images, the latter has been replaced with the differential
attenuation image, which was obtained by calculating the derivative along
the horizontal axis.
\end{quote}

\section*{Reviewer \#3}
We thank Reviewer 3 for his or her critical reading.

Medical X-ray imaging is intrinsically related to dose issues, and dose
strongly depends on the attenuation properties of the patient.
Therefore, it does not sound reasonable to develop X-ray diagnostic
techniques operating at lower energies than those available now, since this will
increase the dose imparted to the patient significantly. This is why at
present, worldwide efforts in
X-ray imaging --- both in CT and in 2D radiography, including
mammography ---
are focused on the reduction of the dose. Most of the players are working on
the development of reconstruction algorithms dealing with fewer
projections, different acquisition schemes or improved detectors. The
contrast mechanism, however, is always the well-known absorption. Our team,
as well as other groups around the world, works on a more fundamental
aspect, trying to obtain contrast  from refraction and scattering phenomena
rather than absorption. Moreover, it is a general established concept that
phase contrast is expected to provide better contrast for equal or smaller dose
at the same X-ray energy. This is now explained in more detail in the
discussion section.

The reviewer argues that
there is no need to increase the energy up to \SI{100}{\kilo\eV} in order to
\reviewer{translate
this technology to clinic} while at the same acknowledging that 
an increase to \SI{60}{\kilo\eV} would be good for mammography. But
\SI{60}{\kilo\eV} is almost twice
the energy that is used today for conventional mammography
investigations. This comment is in contradiction with his or her first
statement but clearly supports and justifies our efforts to push phase-contrast
imaging up to energies around \SI{100}{\kilo\eV} or higher as those are the energies
relevant for medical applications, without considering the mammography
\emph{niche}

 The reviewer also states that efforts devoted to reduce the
 source-to-detector distance would not be supported by industry
 (\reviewer{discouraged industry applications}). Even though we agree that there
 might be several applications --- nondestructive testing and homeland
 security for instance --- where a source-to-detector distance of
 \SI{2}{\metre} would be
 desired, we would like to point out the crucial issue of detector
 efficiency which, combined with the inverse-square law of flux reduction,
 significantly affects the total time one X-ray investigation takes. Time is
 an issue, as patients move and affect the final image quality. Last but not
 least, there is an economic factor to take into account: no hospital will
 buy a system which is ten times slower than the already available technology.

 On this respect, but considering a different aspect of our new technology,
 the reviewer is indeed mentioning the potential bottleneck
 of the edge-on approach, namely the 
 (slow) scanning geometry. The
 main reason for a scanning geometry is that for this proof of principle we just fabricated one set of
 gratings. Obviously, one can consider stacking such gratings and aligning
 them to a multi-line detector following the  principle of spiral CT when
 operated with multiline detectors: this would speed up acquisition
 significantly.
 
 For 2D radiography, the commercially available MicroDose System of
 Philips with a scanning geometry and a strongly collimated beam 
 results in a reduced dose to the patient. It follows that having a scanning
 geometry could result in a lower dose deposition when compared with a 2D
 area illumination, without an impractical increase in exposure time.

 \begin{reviewerquote}
     This is perhaps the reason that in spite of large
     scale effort by industrial leaders in radiology, the phase contrast
     radiology still remains as academic research.
 \end{reviewerquote}
 Since we just mentioned a big player in the medical imaging field, we
 would like to point out our disagreement with this statement. While it is
 true that most of the efforts are still carried out at academic sites,
 industrial activities on phase contrast within companies like Philips,
 General Electrics, Konica-Minolta, Siemens, Zeiss and Brucker are reported
 regularly at international conferences.

Finally, the reviewer raises some doubts about the novelty of the edge-on geometry of our
gratings, by taking examples from
X-ray optics, namely multilayer Laue mirrors. We acknowledge
(also in the manuscript) that previous work reported the usage of tilted
gratings, i.e. Stuntman \emph{et al}.
, but we also clearly explain
 how much our edge-on approach is superior with
respect to previous solutions, mostly because of the unrestricted field of
view given by the matching of the curved wave front. We do not think that the comparison with the Laue
mirrors is fair as the length scales are far from those
related to our experiment. We diffract against micrometer-sized structure
while multilayers are devices with structures in the nanometer-scale.

For these reasons, we cannot share the statements of the reviewer when arguing
about the innovation degree of our work. We believe that high-energy phase
contrast on compact, clinically relevant geometries is achievable with our
approach and therefore we are convinced about the relevance and the
breakthrough character of our results: they show, for the first time,
results at \SI{100}{\kilo\eV} on short distances using a conventional X-ray source.
\end{document}
