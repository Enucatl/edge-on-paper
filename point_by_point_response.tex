%        File: point_by_point_response.tex Created: Wed Oct 23 05:00 PM 2013
%        C Last Change: Wed Oct 23 05:00 PM 2013 C
%
\documentclass[a4paper,english]{scrartcl}
\usepackage[detect-all]{siunitx}
\usepackage{babel}
\usepackage[T1]{fontenc}
\usepackage[utf8]{inputenc}

\DeclareSIUnit{\voltpeak}{Vp} \newcommand{\reviewer}[1]{\emph{#1}}
\newenvironment{reviewerquote}{\begin{quote}\itshape}{\end{quote}}

\title{Response to the reviewer}
\author{Revised manuscript SREP-13-06290}
\date{}
\begin{document}
\maketitle

\noindent

We would like to thank reviewer \#3 for the detailed feedback on our
latest revision. We actually share most of his or her concerns but we also see
that there have been some misunderstanding on our side on the main points
raised by his or her latest comments. We would like to address them in the
following. 

\begin{reviewerquote}
It is disappointing that the authors misinterpreted the
questions and the arguments this referee raised against the major claims of
this work. This referee did not say: "there is no need to increase the
energy to \SI{100}{\kilo\eV} in order to translate this technology to clinic" but
rather question that is not the major factor. This referee did not
"acknowledge(ing) that an increase to \SI{60}{\kilo\eV} would be good for mammography",
even though it could be. The main point of the argument is if the
contribution made by this work important or significant. The conclusion of
the first review was not enough for a journal cover all different areas of
basic science. It is probably not even sufficient to attract to readers in
the same research area of X-ray imaging. The authors reply did not change
this concern by quoting an "improvement" of increasing photon energy to 60
keV for mammography, or to higher photon energy in general for radiology in
general, a well-established and widely implemented for human health care for
decades.

Specifically, this referee questioned if the photon energy (not
being high enough) and the distance (not being short enough) between
detector and specimen, are the key barriers to the translation of grating
base phase contrast radiology to clinic. The major barrier prevent the phase
contrast imaging methods, re-invented more than 20 years ago with the birth
of new synchrotron facility, to be transfer away from synchrotron and to
make an impact to clinical application at this moment, is the long exposure
time required to obtained the phase information by either high quality
sources (in-line) or treatment to the exit waves. Talbot-Lau method promises
phase contrast using conventional X-ray sources by slotting the large source
into small ones which could potentially enable the phase retrieval. However,
the requirement of scanning the gratings, in the case reported here (24
steps), inevitably increases the dose as well as the image acquisition time
which put this technique only in the research laboratory at the moment.
\end{reviewerquote}

Apparently, we have misinterpreted the message of reviewer \#3
and we thank him or her for the effort in clarifying the comments.


While most of the industrial research and product development is obviously
unknown to the general public, at least two product roadmaps have been presented in
the last months. An upgrade of the Philips MicroDose mammography system into a
Talbot-Lau interferometer was shown by the company at the SPIE
2014 conference\cite{Roessl2014,Roessl06032014}. Moreover, clinical trials are currently
ongoing on patients and healthy volunteers in the hospital of the Saitama
Medical University with a Talbot-Lau interferometer
developed as a collaboration between the University of Tohoku and Konica
Minolta~\cite{Momose06032014}. This last machine operates with an
acceleration voltage of~\SI{40}{\kilo\voltpeak} and with an acceptable
exposure time (\SI{19}{\second}) and dose (under \SI{10}{\milli\gray}).

We think that this is convincing evidence that this technique already belongs to the
hospitals and not only to the research laboratories, as well as being a hot
topic of recent industrial and scientific development.

However, the reviewer is right when he points out that the work presented in
the manuscript is still far from applications. At the same time, I think he
or she can agree on the fact that sources
with~\num{100}-\SI{120}{\kilo\voltpeak} are needed in
hospitals for tomography and even higher voltages are required for security
inspections or material sciences investigations.
This is why we think that, besides reduction of dose and exposure times for
existing setups, which was very successful in the last years, the
exploration of a new energy range is also a fundamental
research topic. Indeed, up to now, no system in the world was able to
produce images at energies around~\SI{100}{\kilo\eV}. Even with a high
dose or long exposure
this was simply impossible because of the unavailability of the optical
elements, which we could overcome.

In the first part of the argumentations, reviewer \#3 questions whether our
contribution is ``important of significant''. This is a remark which in
principle can be made to any scientific contribution, independently to its
degree of novelty, difficulty or outreach. It has a strong subjective
component, as demonstrated by the fact that two others reviewers found our
work more than worth to be published in Scientific Reports while reviewer \#3 still has
several concerns. Therefore, we would like to reiterate here that our work
is a step towards the implementation of phase contrast imaging on clinical
systems which we believe might contribute to improve image contrast while
keeping the dose under reasonable control. Those systems are compact and are
operated at high energies, or at least at energies much higher than those
that have been used for any phase-contrast experiment so far to the best of
our knowledge. We are not claiming that we found the ultimate solution to
it, but we present an experimental that addresses to address two major concerns
(high energy and compactness) which are presently limiting the broad
application of phase contrast imaging outside the synchrotron community.

Finally, low-energy grating interferometry is in fact making its appearance
in hospitals in dedicated areas (cartilage and mammography), and developing
high-energy systems is key to expanding its applications, as well as
potentially giving
additional benefits in terms of phase contrast and dose reduction as shown
in the manuscript.
This study is of course a first step toward this goal and,
while we could claim a successful design and operation of the machine, we
cannot conclude at this stage that we can have a high imaging quality with
clinically compatible exposures.

\begin{reviewerquote}
This referee knows and agree that the radiation dose is critical
for medical application, but that was exactly where the usefulness of this
work is quested. It is well accepted that no new X-ray imaging technology is
likely be implemented for clinic with dose penalties. There is even more
fundamental question on if the phase information can really benefit clinical
diagnosis even at the same dose. Instead of tackling the main challenges on
reducing the scanning time or on demonstrating that phase contrast indeed
give better visibility at the same dose than those without phase contrast on
critical applications such as for medical imaging, device inspection and
security inspection.
\end{reviewerquote}

Clinical relevance of phase contrast is a fundamental question
and we addressed this in several works, specifically on mammography,
published elsewhere~\cite{Stampanoni2011,Donath2010a,Wang2014}. We are very well aware
about the importance of assessing the advantages of phase contrast imaging
with respect to existing diagnostic procedures, in particular when dealing
with dose issues. However, the main subject of the present manuscript is
different and addresses a very practical aspect: we will not even be able to
discuss about clinical relevance of phase contrast if we cannot operate
phase contrast in a clinical environment at all. With our work we want to
show a possible solution to implement phase contrast imaging in clinics and
addressing high energies and compact systems seem to us to be two
fundamental points. 

\begin{reviewerquote}
suggested secondary problems, such as photon energy and the
total dimension requirement for the instrument are targeted. The efforts are
trying to "improve", if it is indeed an improvement, a technology (the
grating base phase contrast radiology) that is interesting only for its
promised clinical application which otherwise offers no fundamental
innovation for the phase retrieval scheme in X-ray imaging (it took
advantage of a known theory in optics to shorter wavelength).
\end{reviewerquote}

We do not believe that photon energy and total dimension
requirements are secondary problems. As mentioned before, the penetration
power of the probing radiation is rather a primary issue since we need
sufficient photon statistics to be generated after penetrating a whole human
body, whether we
perform absorption or phase-contrast imaging. Moreover, some clinical
applications are not only promised, but are actually coming true as shown
above.

The comments on whether the phase
retrieval procedure is innovative or not could actually be extended to the
whole field of grating interferometry, as the theory in optics is known
since the 19th century, although at the time it was only applied to longer
wavelengths. We would like to point out again that the promised clinical
applications are coming true, and that the realization of this experiment
had its remarkable challenges, despite being known in theory.

\begin{reviewerquote}
Although Scientific Reports requests evaluation on the
scientific rigorousness, over claiming the potential on clinical, device and
homeland security applications is not rigorous. The hastiness of the
presentation does not make their work rigorous, either. There is no analysis
presented on the only image result, other than just pointing out the
differential phase effect and merging images of different contrast. The
authors argued that the reduced visibility in the attenuation contrast is
due to the stronger X-ray hardening effect than that of phase contrast. This
cannot be correctly evaluated by the referees and readers without knowing
the size (along to the beam direction) of the structure of the chip,
including the solders and resistors, and the absolute gray scale of those
images. Without giving detailed analysis of the contrast on a set of poor
quality images, and perhaps compare them with lower, say \SI{80}{\kilo\eV} photons, it
is not convincing that the work does indeed improve the imaging quality of
visibility for their targeted applications.
\end{reviewerquote}

It is true that the current data are not sufficient to claim an improvement
on imaging quality or sensitivity on details in
general, which is why we do not draw such conclusions in the paper, as the
reviewer points out. However, we can already claim that the only
previously published similar system running on an X-ray tube
at a design energy of~\SI{60}{\kilo\eV}~\cite{:/content/aip/journal/rsi/80/5/10.1063/1.3127712}, also
cited in the manuscript, reported a
visibility of only~\SI{3}{\percent}, hence our system is already performing
much better. We also note once more that comparisons with
synchrotron sources are completely out of the question since the
monochromaticity, brilliance and coherence of third generation synchrotrons
have a great influence on image quality but are impossible to match on a
laboratory source, and cannot obviously be deployed on a large scale.

\begin{reviewerquote}
There are also other flaws of this presentation, which were not
pointed out in the first round reviewer's report. First of all, there is no
mention about the exact photon energy or range actually used. Referring to
manufactures data and provide a few more words than "designed energy of 100
keV" would be more quantitative. The accelerating energy and photon energy
are used in a confusion fashion of kV, kVp and keV for non-specialists. If
the average photon energy is~\SI{100}{\kilo\eV} then the author must demonstrate
significant difference in terms of the image quality between previous
reports of 82keV and~\SI{60}{\kilo\eV} using synchrotron and Talbot-Lau methods. If
photon energy is one of the major claim of advantage of this paper, the
authors cannot escape this comparison by just saying they are done with
synchrotron or by other type of grating arrangements.
\end{reviewerquote}

More details on the source, such as the material of the anode and the
current have been added to the manuscript. The energy range is not changed
afterwards by any means, so this is indeed the full
information needed to reproduce the experiment on any similar generator.
The design energy is a property of the interferometer, and is defined in the
manuscript as the energy at which ``the beam splitter grating periodically
shifts the phase by zero and $\pi$''. We suggest that this is already a
precise statement.

The units for the energy of the
photons and the voltage of the source were checked once more, and the authors could not find any
occurrence where they are confused or incorrectly used, as suggested by the
reviewer. We would of course welcome a more detailed reference to such a mistake, just
to make sure there are no trivial inconsistencies.

Concerning the comparison with synchrotron experiments, we think
that at this stage of the development, it should be sufficient to show that
we can generate phase contrast images at~\SI{100}{\kilo\eV} on a conventional source
and to provide the discussion the way and extent we did. Again, the differences
between a synchrotron and a tube-based experiment are so big that the fact
already that we were able to obtain phase image should be considered as a
success and worth to be
published: up to now, nobody was able to achieve this.
The authors do not agree that this is ``just another type of grating
arrangement'': the traditional method used up to now in grating
interferometers (face-on illumination) simply makes fabricating the aspect
ratios needed for high energies impossible. Moreover, it seems unlikely that
the electroplating techniques can improve by one order of magnitude in the
near future.
Hence, the suggested arrangement this might be the \emph{only} arrangement
that allows grating interferometry on this energy range.

\begin{reviewerquote}
The claim of potential having an order of magnitude decrease in
the scanning time is not supported by their own argument. If it is indeed
limited by the visibility of the grating, how can an improvement from 5\%
(device reported here) to 14\% (current best technology) reaching one order
of magnitude, "easily”? 
\end{reviewerquote}

This is because, as explained in the methods section, some
smaller section of our current gratings already have a visibility
of~\SI{14}{\percent}. 
This means it is almost three times as good as the average, and since the
signal-to-noise ratio drops with the square root of the exposure time, but
proportionally to the visibility, we claim that improving the visibility by
a factor of three allows us to shorten the exposure time by a factor of
nine~\cite{Raupach2011}.
Note that this is not the current best technology
available in theory, but what we already have in the first generation of
gratings.              

\begin{reviewerquote}
Even with one order of magnitude improvement, could they obtain
an image with scanning at a speed fast enough to match the desired
motion-free imaging?
\end{reviewerquote}

The reviewer should agree that if we were already able to do
this, the focus and the cut of our publication would be completely
different. Namely, we would already be where we are aiming at:
phase-contrast imaging at high energies, on compact systems, fast and,
ideally, with the 
same or lower dose. We can only hypothesize that, with the suggested grating
design, there are no indications against reaching the theoretical maximal
visibility one might expect at~\SI{100}{\kilo\eV}. This would be physically the best
result achievable. If, in that case, the hardware around the gratings will
be carefully selected, we do not see why we should not be able to acquire
tomographic slices with the same speed as present, absorption-based systems.

\begin{reviewerquote}
The authors argued (in their reply) that they can increase the
acquisition efficiency by stacking the grating. In this case, there will be
200 such devices needed to be stacked to cover a 2 cm wide area, which they
have to consider the curvature in this direction as well. At the same time
they argued that scanning a small beam (a line illumination in this case) is
better than full field imaging. They quote a commercial system claiming
reduction of dose which is likely, as pointed out by the author, due to the
strongly collimation, therefore, is irrelevant to the comparison.
\end{reviewerquote}

The authors think their argument has been slightly misunderstood in the
above argument: we suggest that stacking several collimated interferometers
may be possible,
while still keeping a scanning procedure, exactly like in the Philips
MicroDose system.
This does not imply covering a \SI{2}{\centi\meter} wide area in one direction,
which we agree would be a challenge. We would rather
have fewer lines and scan longer.

\begin{reviewerquote}
The fabrication of the grating is not described specifically,
other than saying "Grating design and fabrication is nonstandard and
involves a complex mask design" with a reference to a fabrication company.
It is not likely one can follow the actual process with just these
information. 
\end{reviewerquote}

The fabrication through lithography and electroplating is not
the main topic of this manuscript, hence it is only cited in 22, 23, 24. The
mask design was indeed carried out by MicroWorks GmbH, but it is a very technical
detail that we think bears no impact on the reproducibility of the
experiment: any design that reliably builds the described circularly
arranged pattern will produce usable gratings.

\begin{reviewerquote}
The SEM image on a selected portion of a grating already shows a
rather poor quality placement of um size structures, no mention about how
the precision of these structure could affect the imaging or the phase
retrieval.
\end{reviewerquote}

We mention in the manuscript that these irregularities are
responsible for the limited visibility of the interferometer, and the
connection of this value to the signal-to-noise ratio in the
images~\cite{Raupach2011}.

\begin{reviewerquote}
In terms of the innovation in the arrangement the grating in the
"edge-on" fashion, Ref. 25 indeed reported a similar arrangement to overcome
the aspect ratio limitation. As to the adoption of a curved structures, it
was also proposed in the fabrication of multilayer Laue lens to deal with
the similar problem by tilting, wedging or curving the nanostructures.
Therefore, the work indeed puts these ingredients together to realize the
device, but does not reach high innovation level by applying it to a device
at different, actually easier, length scale
\end{reviewerquote}

We do not claim that the
suggested grating arrangement is completely unheard of, and we
provide specific references to similar ideas. The goal is to present
phase-contrast images at~\SI{100}{\kilo\eV}
on a compact system. To the best of our knowledge this is novel and, as
we tried to explain, could have a big impact.

\begin{reviewerquote}
With these observations, this referee argues that this paper
does not meet the high scientific or innovation value of Scientific Reports.
The presentation of the results is not rigorous which is not likely be
corrected without much more investigation to build a better publishable
case. The rebuttal did not address the main concern of the last review,
therefore, this referee does not consider this work after revision
publishable for Scientific Reports.
\end{reviewerquote}

We hope to have addressed in our present reply the
concerns of the reviewer. We tried to clarify where we see our work to be
novel and where we discuss the potential benefits of our approach. While
lower energy phase-contrast systems are now entering hospitals, the
generation of a phase-contrast image at~\SI{100}{\kilo\eV} using a conventional X-ray
tube in a compact setup is something which was never done before and that
intrinsically carries a high potential for medical applications and beyond.
This is our message and we are convinced of its relevance and impact which
seems to be well matched to the requests of Scientific Reports: a journal
which, to the best of our knowledge, has no absolute novelty criterion as a
prerequisite for a manuscript to be eligible for publication.

\bibliography{library}
\bibliographystyle{plain}
\end{document}
