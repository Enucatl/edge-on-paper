%        File: point_by_point_response.tex
%     Created: Wed Oct 23 05:00 PM 2013 C
% Last Change: Wed Oct 23 05:00 PM 2013 C
%
\documentclass[a4paper,english]{scrartcl}
\usepackage[detect-all]{siunitx}
\usepackage{babel}
\usepackage[T1]{fontenc}
\usepackage[utf8]{inputenc}

\DeclareSIUnit{\voltpeak}{Vp}
\newcommand{\reviewer}[1]{\emph{#1}}
\newenvironment{reviewerquote}{\begin{quote}\itshape}{\end{quote}}

\title{Response to reviewer \#3}
\author{Revised manuscript SREP-13-06290}
\date{}
\begin{document}
\maketitle

\noindent

The authors obviously share most of the concerns raised by the reviewer, but
there has been some misunderstanding on our side on the main points raised
with his or her latest comments.
We are confident we can thoroughly address them here.

The most important issue is

\begin{reviewerquote}
    this referee questioned if the photon energy (not being high enough) and
    the distance (not being short enough) between detector and specimen, are
    the key barriers to the translation of grating base phase contrast
    radiology to clinic. The major barrier prevent the phase contrast
    imaging methods, re-invented more than 20 years ago with the birth of
    new synchrotron facility, to be transfer away from synchrotron and to
    make an impact to clinical application at this moment, is the long
    exposure time required to obtained the phase information by either high
    quality sources (in-line) or treatment to the exit waves. Talbot-Lau
    method promises phase contrast using conventional X-ray sources by
    slotting the large source into small ones which could potentially enable
    the phase retrieval. However, the requirement of scanning the gratings,
    in the case reported here (24 steps), inevitably increases the dose as
    well as the image acquisition time which put this technique only in the
    research laboratory at the moment. [\ldots] In the long list of "major
    players" and with all their effort contributing to the international
    conferences, is there any one ready to commit this technology as their
    product roadmaps? This work therefore promises a potential benefit on a
    potential technique which is certainly valuable from the instrumentation
    point of view and not likely to generate much interest of readers of
    Scientific Reports.
\end{reviewerquote}

While most of the industrial research and product development is obviously
unknown to the general public, at least two product roadmaps have been presented in
the last months. An upgrade of the Philips MicroDose mammography system into a
Talbot-Lau interferometer was shown by the company at the SPIE 
2014 conference\cite{Roessl2014}. Moreover, clinical trials are currently
ongoing on patients and healthy volunteers in Japan with a Talbot-Lau interferometer
developed as a collaboration between the University of Tohoku and Konica
Minolta~\cite{Momose06032014}. This last machine operates with an
acceleration voltage of~\SI{40}{\kilo\voltpeak} and with an acceptable
exposure time (\SI{19}{\second}) and dose (under \SI{10}{\milli\gray}).

We think that this is convincing evidence that this technique already belongs to the
hospitals and not only to the research laboratories, as well as being a hot
topic of recent industrial and scientific development.

However, the reviewer is right when he points out that the work presented in
the manuscript is still far from applications. At the same time, I think he
or she can agree on the fact that sources
with~\num{100}-\SI{120}{\kilo\voltpeak} are needed in
hospitals for tomography and even higher voltages are required for security
inspections or material sciences investigations.
This is why we think that, besides reduction of dose and exposure times for
existing setups, the exploration of a new energy range is a fundamental
research topic. Indeed, up to now, no system in the world was able to
produce images at energies around~\SI{100}{\kilo\eV}. Even with a high
dose or long exposure
this was simply impossible because of the unavailability of the optical
elements, which we could overcome.

Finally, low-energy grating interferometry is in fact making its appearance
in hospitals in dedicated areas (cartilage and mammography), and developing
high-energy systems is key to expanding its applications, as well as
potentially giving
additional benefits in terms of phase contrast and dose reduction as shown
in the manuscript.
This study is of course a first step toward this goal and,
while we could claim a successful design and operation of the machine, we
could conclude at this stage that we can have a high imaging quality with
clinically compatible exposures.

\begin{reviewerquote}
     The efforts are trying to "improve", if it is indeed an improvement, a
     technology (the grating base phase contrast radiology) that is
     interesting only for its promised clinical application which otherwise
     offers no fundamental innovation for the phase retrieval scheme in
     X-ray imaging (it took advantage of a known theory in optics to shorter
     wavelength).
\end{reviewerquote}
The same comment applies to the whole field of grating interferometry, as the
theory in optics is known since the 19th century, although at the time it
was only applied to longer wavelengths. We would like to point out again
that the promised clinical applications are coming true, and that the
realization of this experiment had its remarkable challenges, despite being
known in theory.

\begin{reviewerquote}
there is no mention about the exact photon energy or range actually used.
Referring to manufactures data and provide a few more words than ``designed
energy of 100 keV'' would be more quantitative.
\end{reviewerquote}

More details on the source, such as the material of the anode and the
current have been added to the manuscript. The energy range is not changed
afterwards by any means, so this is indeed the full
information needed to reproduce the experiment on any similar generator.
The design energy is a property of the interferometer, and is defined in the
manuscript as the energy at which ``the beam splitter grating periodically
shifts the phase by zero and $\pi$''. We suggest that this is already a
precise statement.

The units for the energy of the
photons were checked once more, and the authors could not find any
occurrence where they are confused or incorrectly used, as suggested by the
reviewer.

\begin{reviewerquote}
    how can an improvement from 5\% (device reported here) to 14\% (current best technology) reaching one order of magnitude, "easily"?
\end{reviewerquote}

This is because, as explained in the methods section and reference 33, some smaller section
of our current gratings already have a visibility of \SI{14}{\percent}. This
means it is almost three times as good as the average, and since the
signal-to-noise ratio drops with the square root of the exposure time, but
proportionally to the visibility, we claim that improving the visibility by
a factor of three allows us to shorten the exposure time by a factor of
nine.

\begin{reviewerquote}
    The fabrication of the grating is not described specifically, other than
    saying ``Grating design and fabrication is nonstandard and involves a
    complex mask design'' with a reference to a fabrication company.
\end{reviewerquote}

The fabrication through lithography and electroplating is not the main topic
of this manuscript, hence it is only cited in 22, 23, 24. The mask design
was carried out by MicroWorks GmbH, but it is a very technical detail that
we think bears no impact on the reproducibility of the experiment: any
design that reliably builds the described circularly arranged pattern will
produce usable gratings.

\begin{reviewerquote}
    The SEM image on a selected portion of a grating already shows a rather
    poor quality placement of \SI{}{\micro\meter} size structures, no mention about
    how the precision of these structure could affect the imaging or the
    phase retrieval.
\end{reviewerquote}

We mention in the manuscript that these irregularities are responsible for the limited
visibility of the interferometer, and the connection of this value to the signal-to-noise
ratio in the images.

\bibliography{library}
\bibliographystyle{plain}
\end{document}
