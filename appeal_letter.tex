\documentclass[a4paper,english]{scrlttr2}
\usepackage[T1]{fontenc}
\usepackage[utf8]{inputenc} %Umlaute
\usepackage{babel}
\linespread{1.05}

\pagestyle{empty}
\setlength{\parskip}{.5\baselineskip}

\date{\today}
\begin{document}
\begin{letter}{Dr. Congcong Huang\\

Associate Editor\\
Nature Communications \\

Unit 10-11, 42F, The Center\\
989 Changle Road, Xuhui District\\
Shanghai, China, 200031}
    \opening{Dear Dr. Huang,}

We refer to your communication of October 23, concerning our appeal to the negative
decision about our manuscript NCOMMS-13-09247-T. We are very happy to grasp that
you are now essentially welcoming a revised version of our paper, provided we also
submit a comprehensive, point-by-point reply to the reviewer’s comments as well as a
letter explaining why we are strongly convinced that our manuscript warrants further
consideration by Nature Communications. Our arguments on this are presented in the
following. We would like to reiterate first the reasons for our appeal and then we would
like to highlight the achievements obtained in our work to justify our claims for further
consideration.

The reason for our appeal was triggered by the motivations raised by the reviewers
against our work, which we found not fully correct and, in some cases, even too
simplistic or simply wrong.

Referee one would actually welcome a revised manuscript, which would address his/her
comments and is not advising a rejection a priori. The mentioned work of Stutman et al.
has an intrinsic limitation (the tilt angle) which our approach does not have. Exactly this
issue prevents the method of Stutman to achieve sufficiently high-aspect ratio and curved
profiles at the same time, something that our method easily can. The problem mentioned
on the contrast below the soldering points can be explained by beam-hardening in the
absorption image. We have amended the manuscript accordingly, providing more
information as requested.

Referee two is skeptical about the visibility achieved in our experiment compared with
the work from Pfeiffer's group. He/she might not have noticed that our data have been
obtained on a commercially available X-ray tube, while the cited experiment of Pfeiffer's
et al. has been carried out at a third generation synchrotron source, where notably
unsurpassed beam conditions can be created which are far from what one can expect in
real life (i.e. with conventional X-ray tubes). And, even if one would like to still carry out
this (unfair) comparison, we should mention that our experiment is the first measurement
of its kind with the first generation of gratings made according to our innovative design.
Pfeiffer's team uses gratings obtained after almost a decade of development and, despite
this and the use of a synchrotron, they are not really further than us while they still carry
the intrinsic limitation of a limited grating height. We are well aware that the quality of
our images has still to be improved, but here we are presenting the first experiments with
edge-on grating illumination: we describe its great potential but also mention its
(temporary) limitations. The same reviewer also complains about the limited field of
view (FOV) our method is supposed to have. This is actually wrong, since the horizontal
FOV is just limited by the size of the wafer (as for other methods) while the vertical FOV
is covered by a vertical scan of the sample, i.e. not limited at all, and, even more
important, it becomes very interesting when CT applications are considered (all modern
CT systems combine a line (or few lines) detector with a patient translation). Reviewer 2
finally admits that in comparison with a coded-aperture based experiment at 100 kVp our
images (taken at 160 kVp) are better, which we of course appreciate. We believe that
novelty in the approach must be the driving factor in judging the suitability of our work
for this journal and not the present limitations of the technology (which can be solved!),
as reviewer 2 seems to conclude.

In our opinion, referee three weighed the technicalities and related limitations of our
experiment too much, without considering the innovative solution that we are
introducing. As mentioned above, we are perfectly aware of the constraints of the present
system but here we are showing the feasibility of a novel approach, which of course
needs to be optimized to reach a broader usage. He/she mentions, as comparison, an ABI
experiment performed at the tungsten 60 keV $K_\alpha$ line. We appreciate this hint and
will be happy to integrate it in our revised discussion but we would like to already point
out that our experiment was carried out at 160 kVp, with a nominal energy around 100
keV. This is significantly higher than 60 keV. We are not aware of any ABI setup
operated at 100 keV on a conventional X-ray source, so the comparison does not sound
very fair either.

Finally, we would like to highlight the pioneering character of our work. All the gratingsbased
X-ray phase contrast papers published in the past decade report on experiments
carried out in the face-on-configuration or light modifications of such. We are the first
team demonstrating that an edge-on approach can be successfully implemented and we
clearly describe the advantages of this novel approach compared to the status quo. We are
firmly convinced that the potential of our method is huge, since it can solve all the open
issues (high-energy imaging, no intrinsic aspect ratio limitation in beam-direction,
acceptance of spherical wave profile, scanning and spiral-CT compatible geometry just to
cite a few) which have been preventing gratings-interferometry to establish itself on
commercial, table-top X-ray systems so far.

For the above-mentioned reasons, we are kindly asking you to consider our improved,
revised manuscript for publication in Nature Communications.

\closing{On behalf of the authors,}

M. Stampanoni
\end{letter}
\end{document}
