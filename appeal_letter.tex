\documentclass[a4paper,english]{scrlttr2}
\usepackage[T1]{fontenc}
\usepackage[utf8]{inputenc} %Umlaute
\usepackage{babel}

\pagestyle{empty}

\begin{document}
\begin{letter}{Dr. Congcong Huang\\

Associate Editor\\
Nature Communications \\

Unit 10-11, 42F, The Center\\
989 Changle Road, Xuhui District\\
Shanghai, China, 200031}
    \opening{Dear Dr. Huang,}

We have recently received a negative decision on our
manuscript \textsc{NCOMMS-13-09247-T}. We would like to formally appeal this
decision and ask for the permission to submit a revised version of our
manuscript addressing the concerns of the reviewers. The reason for our
appeal is mainly driven by the motivations raised by the reviewers against
our work which we found not fully correct and, in some cases, even too
simplistic. For your convenience, we will briefly summarize our arguments in
the following lines.

Referee one would actually welcome a revised manuscript, addressing his/her
comments and is not advising a rejection a priori. The mentioned work of
Stutman et al has an intrinsic limitation (the tilt angle) which our
approach does not have. Exactly this issue prevents the methods of Stutman
to achieve sufficiently high-aspect ratio \emph{and} curved profiles at the same
time, something that our method easily can. The problem mentioned on the
contrast below the soldering points can be explained by beam-hardening in
the absorption image. We would amend the manuscript accordingly, providing
more information as requested.

Referee two is skeptical about the visibility achieved in our experiment
compared to the work from Pfeiffer's group. He/she might not have noticed
that our data have been obtained on a commercially available X-ray tube,
while the cited experiment of Pfeiffer's et al. has been carried out at a
    third generation synchrotron source, where notably unsurpassed beam
    conditions can be created which are far from what one can expect in real
    life (i.e. with conventional X-ray tubes). And, even if one would like
    to still carry out this (unfair) comparison, we should mention that our
    experiment is the \emph{first} measurement of its kind with the first
    generation of gratings made according to our innovative design.
    Pfeiffer's team uses gratings obtained after almost a decade of
    development and, despite this and the use of a synchrotron, they are not
    much further than us and still carry the intrinsic limitation of a
    limited grating height. We are perfectly aware that the quality of our
    images has still to be improved, but here we are presenting the first
    experiments with edge-on grating illumination, describing its great
    potential but also acknowledging its (temporary) limitations. The same
    reviewer also complain about the limited field-of-view and a potentially higher
    dose that our method would generate compared to previous solutions. Both
    arguments are wrong, since the horizontal field-of-view is just limited by the
    size of the wafer (as for other methods) while the vertical
    field-of-view is
    covered by a vertical scan of the sample, i.e. not limited at all, and,
    more important, it does not require a higher dose. Reviewer 2 finally
    admits that in comparison with a coded-aperture based experiment at 100
    kVp our images (taken at 160 kVp) are better, which we of course
    appreciate. We believe that novelty in the approach must be the driving
    factor in judging the suitability of our work for this journal and not
    the present limitations of the technology (which can be solved!), as
    reviewer 2 seems to conclude.

In our opinion, referee three weighed the technicalities and related
limitations of our experiment too much, without considering the innovative
solution that we are introducing. As mentioned above, we are perfectly aware
of the constraints of the present system but here we are showing the
feasibility of a novel approach, which of course needs to be optimized to
reach a broader usage. He/she mentions, as comparison, an ABI experiment
performed at the tungsten 60 keV $K_\alpha$ line. We appreciate this hint and
will be happy to integrate it in our revised discussion but we would like to
already point our that our experiment was carried out at 160 kVp, with a
nominal energy around 100 keV. This is significantly higher than 60 keV. We
are not aware of any ABI setup operated at 100 keV on a conventional X-ray
source, so the comparison does not sounds very fair either.

For the above mentioned reasons, we are gently asking you to reconsider our
manuscript and to give us a chance to resubmit a revised version.

\closing{On behalf of the authors,}

M. Stampanoni
\end{letter}
\end{document}
